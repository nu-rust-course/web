% vim: ts=2
\documentclass{beamer}

\usepackage{solarslides}
\usepackage{graphicx}
\usepackage{tikz}

\begin{document}

\begin{frame}
\thispagestyle{empty}\centering
\ti{\alert{Mutual Exclusion}}
\hd{EECS 3/495 “Rust”}
\hd{Spring 2017}
\end{frame}

\begin{frame}{Definitions: Time}
  \begin{itemize}
    \item<1-> A thread $A$ has \emph{atomic} events $a_0, a_1, a_2, \ldots$
    \item<2-> Event $a_i^k$ is the $k$th occurrence of event $a_i$
    \item<3-> $a \to b$ means event $a$ \emph{precedes} event $b$
      \begin{itemize}
        \item $(\to)$ is a total order on events
      \end{itemize}
    \item<4-> An \emph{interval} $I_A = (a_0, a_1)$ is the duration between
      $a_0$ and $a_1$
    \item<5-> Interval $I_A^k$ is the $k$th occurrence of interval $I_A$
    \item<6-> $(a_0, a_1) \to (b_0, b_1)$ means that $a_1 \to b_0$
      \begin{itemize}
        \item $(\to)$ is a partial order on intervals
      \end{itemize}
  \end{itemize}
\end{frame}

\begin{frame}[fragile]{A counter class}{}
  ««
  |K:class |D‹Counter›
  {
  |K‹public›:
  	|+|T:int get_and_inc()
    {
    	|+|T:int old = count_;
      count_ = old + |L1;
      |K:return old; |-
    } |-

  |K‹private›:
  	|T:int count_ = |L0;
  };
  »»
\end{frame}

\begin{frame}[fragile]{A problem}{}
««
|T:Counter c;

|T:std::thread t1([|K&]() { c.get_and_inc(); });
|T:std::thread t2([|K&]() { c.get_and_inc(); });

t1.join();
t2.join();

|M‹CHECK_EQUAL›(|L2, c.get_and_inc());
»»
\end{frame}

\begin{frame}[fragile]{Counter class has a critical section}{}
  ««
  |K:class |D‹Counter›
  {
  |K‹public›:
  	|+|T:int get_and_inc()
    {
    	|+|T:int old = count_; |qquad|=|C‹// danger begins›
      count_ = old + |L1;          |>|C‹// danger ends›
      |K:return old; |-
    } |-

  |K‹private›:
  	|T:int count_ = |L0;
  };
  »»
\end{frame}

\begin{frame}{Mutual exclusion}
  We need \emph{mutual exclusion}!

  \pause\medskip
  Definition: Critical sections can't overlap

  \medskip
  More formally, for threads $A$, $B$, and integers $j$ and $k$, either
  $CS_A^j \to CS_B^k$ or $CS_B^k \to CS_A^j$.
\end{frame}

\begin{frame}[fragile]{Solution: A lock (a/k/a mutex)}{}
  ««
  |K‹class› |D‹ILock›
  {
  	|+|K:virtual |T:void unlock() |==|kill%
    |K:virtual |T:void lock() |> |Q‹= 0›;
    |K:virtual |T:void unlock() |> |Q‹= 0›; |-
  };
  |pause
  |K‹class› |D‹Lock› : |K‹public› |D‹ILock› { |ensuremath|cdots };
  »»
\end{frame}

\begin{frame}[fragile]{Using a lock}{}
  ««
  |K:class |D‹Counter›
  {
  |K‹public›:
  	|+|T:int get_and_inc()
    {
    	|+lock_.lock();
      |T:int old = count_;
      count_ = old + |L1;
      lock_.unlock();
      |K:return old; |-
    } |-

  |K‹private›:
  	|T:int count_ = |L0;
  	|T:Lock lock_;
  };
  »»
\end{frame}

\begin{frame}[fragile]{RAII: Resource Acquisition Is Initialization}{}
  ««
  |K:class |D‹Lock_guard›
  {
  |K‹public›:
  	|+Lock_guard(|T‹ILock› |K& lock) : lock_(lock)
    {
    	lock_.lock();
    }

    |ensuremath|sim Lock_guard()
    {
    	lock_.unlock();
    } |-

  |K‹private›:
  	|T‹ILock› |K& lock_;
  };
  »»
\end{frame}

\begin{frame}[fragile]{Using a lock—RAII-style}{}
  ««
  |K:class |D‹Counter›
  {
  |K‹public›:
  	|+|T:int get_and_inc()
    {
    	|+|T:Lock_guard guard(lock_);
      |T:int old = count_;
      count_ = old + |L1;
      |K:return old; |-
    } |-

  |K‹private›:
  	|T:int count_ = |L0;
  	|T:Lock lock_;
  };
  »»
\end{frame}

\begin{frame}{How to implement the lock?}{}
  Two-thread solutions first, then $n$-thread solutions
\end{frame}

\begin{frame}[fragile]{Base class for two-thread lock}{}
  ««
  |K:class |D:Lock_base : |K:public |D‹ILock›
  {
  	|+|C‹// i() is this thread:›
    |T:thread::id i() |T‹const›
    {
    	|K:return |D‹this_thread›::get_id();
    }

    |C‹// j() is the other thread:›
    |T:thread::id j() |T‹const›
    {
    	|K:return i().other_thread();
    } |-
  };
  »»
\end{frame}

\begin{frame}[fragile]{An attempt}{}
  ««
  |K‹class› |D‹Lock_one› : |K‹public› |D‹Lock_base›
  {
  	|T‹bool› flag_|T‹[2]› = {|L‹false›, |L‹false›};

  |K‹public›:
  	|+|K:virtual |T:void lock() |Q‹override›
    {
    	|+flag_[i()] = |L‹true›;
      |K:while (flag_[j()]) {} |-
    }

    |K:virtual |T:void unlock() |Q‹override›
    { flag_[i()] = |L‹false›; } |-
  };
  »»
\end{frame}

\begin{frame}[fragile]{Theorem}{}
  «|D‹Lock_one›» satisfies mutual exclusion.
  \pause \alert{Proof by contradiction:}

  \begin{itemize}
    \item Assume CS$_A$ overlaps CS$_B$
    \pause
    \item Consider each thread’s last read and write in \emph{lock}() before
      entering its CS. For $A$ to enter, it first writes true to its flag,
      and then needs to read false from the other’s:
      \begin{itemize}
        \item write$_A$(flag[$A$] = true) $\to$ read$_A$(flag[$B$] == false) $\to$ CS$_A$
      \end{itemize}
      \pause
      And by symmetry:
      \begin{itemize}
        \item write$_B$(flag[$B$] = true) $\to$ read$_B$(flag[$A$] == false) $\to$ CS$_B$
      \end{itemize}
    \pause
    \item Note, also, that if $A$ sees $B$’s flag as false, that must happen
      before $B$ writes its flag, and by symmetry for $B$ seeing $A$’s flag:
      \begin{itemize}
        \item read$_A$(flag[$B$] == false) $\to$ write$_B$(flag[$B$] = true)
        \item read$_B$(flag[$A$] == false) $\to$ write$_A$(flag[$A$] = true)
      \end{itemize}
      \pause
      These events form a cycle, which is a contradiction.\hfill\qed
  \end{itemize}
\end{frame}

\begin{frame}{Two other properties}{}
  Deadlock-free:
  \begin{itemize}
    \item When threads try to acquire the lock, at least one succeeds
    \item System as a whole makes progress
    \item<2> Does \textT{Lock\_one} enjoy deadlock freedom?
  \end{itemize}
   Starvation-free
  \begin{itemize}
    \item Every locking thread eventually returns
    \item Every thread makes progress
    \item<2> Does \textT{Lock\_one} enjoy starvation freedom?
  \end{itemize}
\end{frame}

\begin{frame}[fragile]{Deadlock case for Lock\_one}{}
  ««
  flag_[0] = |L‹true›;
  |hspace‹.45|linewidth› |=flag_[1] = |L‹true›;
  |K:while (flag_[1]) {}
                         |>|K:while (flag_[0]) {}
  »»
  \bigskip
(But sequentially it's fine.)
\end{frame}

\begin{frame}[fragile]{Another attempt}{}
  ««
  |K‹class› |D‹Lock_two› : |K‹public› |D‹Lock_base›
  {
  	|T‹int› waiting_;

  |K‹public›:
  	|+|K:virtual |T:void lock() |Q‹override›
    {
    	|+waiting_ = i();
      |K:while (waiting_ == i()) {} |-
    }

    |K:virtual |T:void unlock() |Q‹override› {} |-
  };
  »»
\end{frame}

\begin{frame}{Lock\_two claims}{}
  \begin{itemize}
    \item Satisfies mutual exclusion
    \item Non deadlock-free
      \begin{itemize}
        \item Sequential execution deadlocks
        \item Concurrent execution does not
      \end{itemize}
  \end{itemize}
\end{frame}

\begin{frame}[fragile]{Peterson's algorithm}{}
  ««
  |K‹class› |D‹Peterson_lock› : |K‹public› |D‹Lock_base›
  {
  	|T‹bool› flag_|T‹[2]› = {};
  	|T‹int› waiting_;

  |K‹public›:
  	|+|K:virtual |T:void lock() |Q‹override›
    {
    	|+flag_[i()] = |L‹true›;
      waiting_ = i();
      |K:while (flag_[j()] |K‹&&› waiting_ == i()) {} |-
    }

    |K:virtual |T:void unlock() |Q‹override›
    { flag_[i()] = |L‹false›; } |-
  };
  »»
\end{frame}

\begin{frame}{Peterson's lock properties}
  \begin{itemize}
    \item Mutual exclusion
      \begin{itemize}
        \item By contradiction\ldots
      \end{itemize}
    \item Deadlock freedom
      \begin{itemize}
        \item Only one thread at a time can be waiting
      \end{itemize}
    \item Starvation freedom
      \begin{itemize}
        \item If A finishes and tries to re-enter while B is waiting, B gets
          in first
      \end{itemize}
  \end{itemize}
\end{frame}

\begin{frame}[fragile]{Filter algorithm for $n$ threads}{}
  ««
  |K‹template› <|T‹int› N>
  |K‹class› |D‹Filter_lock› : |K‹public› |D‹Lock_base›
  {
  	|+|kill%
    |T‹int› level_|T‹[N]› = {};
    |T‹int› waiting_|T‹[N]›;

    |T:bool exists_competition(|T:int level)
    {
    	|+|kill%
      |K:for (|T:auto k : |D‹thread›::all_ids())
      	|K:if (k != i() |K‹&&› level_[k] >= level)
      		|K‹return› |L‹true›;
      |K:return |L‹false›; |-
    }
    $|vdots$ |-
  }
  »»
\end{frame}

\begin{frame}[fragile]{}{}
  ««
  |K‹template› <|T‹int› N>
  |K‹class› |D‹Filter_lock› : |K‹public› |D‹Lock_base›
  {
  $|vdots$
  |K‹public›:
  	|+|kill%
    |K‹virtual› |T‹void› lock() |Q‹override›
    {
    	|+|kill%
      |K‹for› (|T:int level = |L1; level < N; ++level) {
      	|+|kill%
        waiting_[level] |==|kill%
        level_[i()] |> = level;
        waiting_[level] = i();
        |K:while (|=exists_competition(level) |K‹&&›
                  |>waiting_[level] == i()) {} |-
      } |-
    }
    
    |K‹virtual› |T:void unlock() |Q‹override›
    { level_[i()] = |L0; } |-
  }
  »»
\end{frame}

\begin{frame}{Filter lock properties}
  \begin{itemize}
    \item Mutual exclusion
      \begin{itemize}
        \item By induction, one thread gets stuck in each level...
      \end{itemize}
    \item Deadlock freedom
      \begin{itemize}
        \item Like Peterson—only one thread can wait per level
      \end{itemize}
    \item Starvation freedom
      \begin{itemize}
        \item Like Peterson—every thread advances if any does
      \end{itemize}
  \end{itemize}
\end{frame}

\begin{frame}{}{}
\vskip4pt
\parskip=4pt
\scriptsize
This work is licensed under a Creative Commons “Attribution-ShareAlike
3.0 Unported” license.

These slides are derived from the companion slides for \emph{The Art of
Multiprocessor Programming,} by Maurice Herlihy and Nir Shavit. Its
original license reads:
\begin{quote}
  This work is licensed under a \emph{Creative Commons Attribution-ShareAlike
  2.5 License.}
\begin{itemize}
  \item \textbf{You are free}:
    \begin{itemize}
      \item\tiny \textbf{to Share} — to copy, distribute and transmit the work
        \item\tiny \textbf{to Remix} — to adapt the work
    \end{itemize}
  \item \textbf{Under the following conditions:}
    \begin{itemize}
      \item\tiny \textbf{Attribution.} You must attribute the work to “The Art of
        Multiprocessor Programming” (but not in any way that suggests
        that the authors of that work or this endorse you or your use of
        the work).
      \item\tiny \textbf{Share Alike.} If you alter, transform, or build upon this work,
        you may distribute the resulting work only under the same,
        similar or a compatible license.
    \end{itemize}
  \item For any reuse or distribution, you must make clear to others the
    license terms of this work. The best way to do this is with a link to
    \begin{itemize}
      \item\tiny \url{http://creativecommons.org/licenses/by-sa/3.0/}.
    \end{itemize}
  \item Any of the above conditions can be waived if you get permission from
    the copyright holder.
  \item Nothing in this license impairs or restricts the author’s moral
    rights.
\end{itemize}
\end{quote}
\end{frame}

\end{document}
