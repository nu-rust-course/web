% vim: ts=2
\documentclass{beamer}

\usepackage{solarslides}
\usepackage{graphicx}
\usepackage{tikz}

\begin{document}

\begin{frame}
\thispagestyle{empty}\centering
\ti{\alert{Concurrent Objects and Linearizability}}
\hd{EECS 3/495 “Rust”}
\hd{Spring 2017}
\end{frame}

\begin{frame}{What is a concurrent object?}{}
  \begin{itemize}
    \item<1-2> How do we \alert{describe} one?
    \item<1> How do we \alert{implement} one?
    \item<1-2> How do we \alert{tell if we’re right}?
  \end{itemize}
\end{frame}

\begin{frame}[fragile]{Case study: FIFO queue}{}
  \begin{center}
    q = \begin{tabular}{|c|c|c|c}
      \hline
      \uncover<-4>{{\textL2}} & {\textL4} & \uncover<3->{{\textL6}} & \\
      \hline
    \end{tabular}
  \end{center}
  \uncover<2->{«q.enq({\textL 6})»}

  \uncover<4->{«q.deq()»} \uncover<5->{$\Rightarrow$ \textL2}
\end{frame}

\begin{frame}[fragile]{Implementation: Lock-based ring buffer}{}
  ««
  |P‹|#include <array>›

  |K:template <|K:typename Element, |T:int capacity>
  |K:class |D‹Lock_based_FIFO›
  {
  |K‹public›:
  	|+|T:void enq(|T‹Element›);
    |T:Element deq(); |-

  |K‹private›:
  	|+|T‹std::array<Element, capacity>› data_;
    |T:unsigned head_ = |L0, tail_ = |L0;
    |T:Lock lock_; |-
  };
  »»
\end{frame}

\begin{frame}[fragile]{Implementation: Lock-based enqueue}{}
  ««
  |K:template <|K:typename Element, |T:int capacity>
  |T:void Lock_based_FIFO::enq(|T:Element x)
  {
  	|+|T:LockGuard guard(lock_);

    |K:if (tail_ - head_ == capacity) |K:throw |D‹fifo_full›();

    data_[tail_++ |% capacity] = x; |-
  }
  »»
\end{frame}

\begin{frame}[fragile]{Implementation: Lock-based deque}{}
  ««
  |K:template <|K:typename Element, |T:int capacity>
  |T:Element Lock_based_FIFO::deq()
  {
  	|+|T:LockGuard guard(lock_);

    |K:if (tail_ == head_) |K:throw |D‹fifo_empty›();

    |K:return data_[head_++ |% capacity]; |-
  }
  »»
\end{frame}

\begin{frame}{Now consider this}{}
  Same thing, but:
  \begin{itemize}
    \item no mutual exclusion
    \item only two threads:
      \begin{itemize}
        \item one only enqueues
        \item one only dequeues
      \end{itemize}
  \end{itemize}
\end{frame}

\begin{frame}[fragile]{Wait-free SRSW FIFO}{}
  ««
  |P‹|#include <array>›
  |P‹|#include <atomic>›

  |K:template <|K:typename Element, |T:int capacity>
  |K:class |D‹Wf_SRSW_FIFO›
  {
  |K‹public›:
  	|+|T:void enq(|T‹Element›);
    |T:Element deq(); |-

  |K‹private›:
  	|+|T‹std::array<Element, capacity>› data_;
    |T‹std::atomic<unsigned long>› head_{|L0}, tail_{|L0}; |-
  };
  »»
\end{frame}

\begin{frame}[fragile]{Wait-free SRSW enqueue}{}
  ««
  |K:template <|K:typename Element, |T:int capacity>
  |T:void |D‹Wf_SRSW_FIFO<Element, capacity>›::enq(|T:Element x)
  {
    	|+|K:if (tail_ - head_ == capacity) |K:throw |D‹fifo_full›();

    data_[tail_ |% capacity] = x;
    ++tail_; |-
  }
  »»
\end{frame}

\begin{frame}[fragile]{Wait-free SRSW deque}{}
  ««
  |K:template <|K:typename Element, |T:int capacity>
  |T:Element |D‹Wf_SRSW_FIFO<Element, capacity>›::deq()
  {
    	|+|K:if (tail_ == head_) |K:throw |D‹fifo_empty›();

    |T:Element result = data_[head_ |% capacity];
    ++head_;
    |K:return result; |-
  }
  »»
\end{frame}
\begin{frame}{}{}
\vskip4pt
\parskip=4pt
\scriptsize
This work is licensed under a Creative Commons “Attribution-ShareAlike
3.0 Unported” license.

These slides are derived from the companion slides for \emph{The Art of
Multiprocessor Programming,} by Maurice Herlihy and Nir Shavit. Its
original license reads:
\begin{quote}
  This work is licensed under a \emph{Creative Commons Attribution-ShareAlike
  2.5 License.}
\begin{itemize}
  \item \textbf{You are free}:
    \begin{itemize}
      \item\tiny \textbf{to Share} — to copy, distribute and transmit the work
        \item\tiny \textbf{to Remix} — to adapt the work
    \end{itemize}
  \item \textbf{Under the following conditions:}
    \begin{itemize}
      \item\tiny \textbf{Attribution.} You must attribute the work to “The Art of
        Multiprocessor Programming” (but not in any way that suggests
        that the authors of that work or this endorse you or your use of
        the work).
      \item\tiny \textbf{Share Alike.} If you alter, transform, or build upon this work,
        you may distribute the resulting work only under the same,
        similar or a compatible license.
    \end{itemize}
  \item For any reuse or distribution, you must make clear to others the
    license terms of this work. The best way to do this is with a link to
    \begin{itemize}
      \item\tiny \url{http://creativecommons.org/licenses/by-sa/3.0/}.
    \end{itemize}
  \item Any of the above conditions can be waived if you get permission from
    the copyright holder.
  \item Nothing in this license impairs or restricts the author’s moral
    rights.
\end{itemize}
\end{quote}
\end{frame}

\end{document}
