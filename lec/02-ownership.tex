% vim: ts=2
\documentclass{beamer}

\usepackage{solarslides}
\usepackage{graphicx}
\usepackage{tikz}

\begin{document}

\begin{frame}
\thispagestyle{empty}\centering
\ti{\alert{Ownership and Borrowing \\ and Lifetimes (Oh My!)}}
\hd{EECS 3/495 “Rust”}
\hd{Spring 2017}
\end{frame}

\begin{frame}{Definitions}{}
  An \emph{object} is a chunk of memory with a type
  \par
  Examples:
  \begin{itemize}
    \item The number 4 is a \emph{value}, not an object
    \item A word of memory containing the number 4 is an object
  \end{itemize}
  A \emph{variable} is the name of an object
\end{frame}

\begin{frame}{Ownership}{}
  Every object in Rust has an owner. Either:
  \begin{itemize}
    \item a variable, or
    \item some other object
  \end{itemize}
  \par\pause\medskip
  Ownership comes with rights and responsibilities:
  \begin{itemize}
    \item The owner is allowed to modify the object
    \item The owner must free the object (or pass it to another owner)
  \end{itemize}
\end{frame}

\begin{frame}[fragile]{Transferring ownership}{}
  Ownership can be transferred:
  ««
  |K‹pub fn› inc_vec(|T‹mut› v: |T‹Vec<usize>›, ix: |T‹usize›) {
    	v[ix] += |L1;
  }
  »»
  \pause
  ««
  |P‹|#[test]›
  |K:fn test_inc_vec() {
  	|+|K:let expected |== |M‹vec!›[ 3, 4, 6 ];
    |K:let actual   |>= |M‹vec!›[ 3, 4, 5 ];

    inc_vec(actual, |L2);

    |M‹assert_eq!›(expected, actual); |pause|C‹// Error! actual has been moved› |-
  }
  »»
\end{frame}

\begin{frame}[fragile]{One solution: FP style}
  ««
  |K‹pub fn› inc_vec(|T‹mut› v: |T‹Vec<usize>›, ix: |T‹usize›) -> |T‹Vec<usize>› {
    	v[ix] += |L1;
    	v
  }

  |P‹|#[test]›
  |K:fn test_inc_vec() {
  	|+let mut |=expected |==|kill%
    |K:let |>expected     |>= |M‹vec!›[ 3, 4, 6 ];
    |K:let |T‹mut› |>actual |>= |M‹vec!›[ 3, 4, 5 ];

    actual = inc_vec(actual, |L2);

    |M‹assert_eq!›(expected, actual); |-
  }
  »»
\end{frame}

\begin{frame}[fragile]{The Rust solution: borrowing}
  ««
  |K‹pub fn› inc_vec(v: |T‹&mut Vec<usize>›, ix: |T‹usize›) {
    	v[ix] += |L1;
  }

  |P‹|#[test]›
  |K:fn test_inc_vec() {
  	|+let mut |=expected |==|kill%
    |K:let |>expected     |>= |M‹vec!›[ 3, 4, 6 ];
    |K:let |T‹mut› |>actual |>= |M‹vec!›[ 3, 4, 5 ];

    inc_vec(|K‹&mut› actual, |L2);

    |M‹assert_eq!›(expected, actual); |-
  }
  »»
\end{frame}

\begin{frame}[fragile]{More idiomatic Rust: take a slice}
  ««
  |K‹pub fn› inc_vec(v: |T‹&mut [usize]›, ix: |T‹usize›) {
    	v[ix] += |L1;
  }

  |P‹|#[test]›
  |K:fn test_inc_vec() {
  	|+let mut |=expected |==|kill%
    |K:let |>expected     |>= |M‹vec!›[ 3, 4, 6 ];
    |K:let |T‹mut› |>actual |>= |M‹vec!›[ 3, 4, 5 ];

    inc_vec(actual.as_mut_slice(), |L2);

    |M‹assert_eq!›(expected, actual); |-
  }
  »»
\end{frame}

\begin{frame}[fragile]{Borrowing implements reader/writer semantics}{}
  You can borrow
  \begin{itemize}
    \item as many immutable references as you like, or
    \item one mutable reference.
  \end{itemize}
  ««
  |K:let |T:mut x = |T‹SomeObject›::new();

  {
  	|+|K:let r1 = |K&x;
      |K:let r2 = |K&x;
      |K:let r3 = r1;
      |K:let r4 = |K‹&mut› x; |qquad|=|C‹// error!› |-
  }

  {
  	|+|K:let r5 = |K‹&mut› x;  |>|C‹// ok›
      |K:let r6 = |K&x;        |>|C‹// error!› |-
  }
  »»
\end{frame}

\begin{frame}[fragile]{Hidden borrows}{}
  Methods calls may (mutable) borrow «|D‹self›»:

  ««
  |K:impl |T:SomeObject {
  	|K‹pub fn› f(|D‹&mut self›) { |ensuremath|cdots }
  }



  |K:let x = |T‹SomeObject›::new();

  x.f(); |qquad|C‹// error: x isn't mutable›
  »»
\end{frame}

\begin{frame}{When borrowing won't do}{}
  \begin{itemize}
    \item The \textT{Copy} trait for cheap copies
    \item The \textT{Clone} trait for expensive copies
  \end{itemize}
\end{frame}

\begin{frame}[fragile]{The Copy trait}{}
  Types implementing the «|T‹Copy›» trait are copied implicitly rather
  than moved:
  \begin{itemize}
    \item «|T‹usize›» and other built-in numeric types
    \item «|T‹&str›» and other immutable reference types
    \item In general, types that
      \begin{itemize}
        \item are cheap to copy (small), and
        \item don't involve a \emph{resource} (\emph{e.g., heap
          allocations})
      \end{itemize}
  \end{itemize}

  ««
  |K:let a = |L5;
  |K:let b = a;
  f(a);
  |K:let c = a + b;
  »»
\end{frame}

\begin{frame}[fragile]{The Clone trait}{}
  The «|T‹Clone›» trait supports explicity copying:
  \begin{itemize}
    \item «|T‹String›», «|T‹Vec›», «|T‹HashMap›», etc.
    \item In general, types that
      \begin{itemize}
        \item may be expensive to copy, and
        \item don't involve a \emph{unique resource}
          (\emph{e.g.,} a file handle)
      \end{itemize}
  \end{itemize}

  ««
  |K:let v = |M‹vec!›[ 3, 4, 5 ];
  |K:let u = v.clone();
  f(v);
  g(u);
  »»
\end{frame}

\begin{frame}[fragile]{Lifetimes}
  Object have lifetimes (or more precisely, death times)

  ««
  {
  	|+|K:let |T:mut r: |T‹&str›;

    {
    	|+|K:let s = |T‹String›::new();

      r = |K&s; |qquad|C‹// error because r outlives s› |-
    } |qquad|C‹// s dies here› |-
  } |qquad|C‹// r dies here›
  »»

  \pause
  A reference must die before its referent!
\end{frame}

\begin{frame}[fragile]{The static lifetime}{}
  The only named lifetime is «|T‹'static›»—the lifetime of the whole
  program

  String slice literals have static lifetime. That is,

  ««
  |K:let s: |T:&str = |L‹"hello"›;
  »»

  means

  ««
  |K:let s: |T‹&'static str› = |L‹"hello"›;
  »»
\end{frame}

\begin{frame}[fragile]{Lifetime variables}{}
  Other lifetimes are relative:
  ««
  |K:fn choose<|T‹'a›>(x: |T‹&'a usize›, y: |T‹&'a usize›) -> |T‹&'a usize›|pause {
    	|+|K:if is_even(|K*x) {x}
      |K:else |K:if is_even(|K*y) {y}
      |K:else {|K&|L0} |-
  }
  »»
\end{frame}

\end{document}
